\documentclass{article}

%\usepackage{color}
\usepackage[dvipsnames]{xcolor}
\usepackage{array}
\usepackage{adjustbox}
\usepackage{multirow}

\title{Meeting with Dom 20210322, and subsequent work}
\author{Lena Morrill}

\setlength{\textwidth}{6.7in}
\setlength{\oddsidemargin}{-0.1in}
\setlength{\textheight}{8.6in}
\setlength{\topmargin}{-0.4in}

\def\checkmark{\tikz\fill[scale=0.4](0,.35) -- (.25,0) -- (1,.7) -- (.25,.15) -- cycle;} 
\newcolumntype{L}[1]{>{\raggedright\let\newline\\\arraybackslash\hspace{0pt}}m{#1}}
\newcolumntype{C}[1]{>{\centering\let\newline\\\arraybackslash\hspace{0pt}}m{#1}}
\newcolumntype{R}[1]{>{\raggedleft\let\newline\\\arraybackslash\hspace{0pt}}m{#1}}

\newcommand{\hgh}{{\color{red} H}} % high: bad [bias]
\newcommand{\lw}{{\color{red} L}} % low: bad [bias]
\newcommand{\gd}{{\color{ForestGreen} G}} % good [bias]

\begin{document}

\maketitle

\tableofcontents

\section{Convergence and optimisers}

\paragraph{Fixed effects DM} When removing the coefficient of 100- for the DM likelihood, I get convergence. Otherwise the gradient goes to 0 and I get NA for every single standard error of the parameters.

\paragraph{Diag RE effects DM} kidney renal clear cell RCC now converges (it didn't use to) when removing the 1000 coefficient, and using nlminb.

\paragraph{Full RE effects DM} Removing the coefficient of 1000, still no convergence for kidney renal clear cell RCC, even if increasing number of iterations, and using nlminb, but I don't get the message about gradient too steep.

When using a high value for the initial log lambda (so that it's closer to a multinomial), it doesn't converge either.

When initialising the covariances around 0, and the u at 0, I get that the Hessian of fixed effects was not positive definite., but no NAs.

\paragraph{Not including covariances for signatures with many zero exposures} That gives good convergences!

\clearpage
\section{Bias and coverage}

The beta intercept estimates are biased (to high values) in detriment of the standard deviation of the random effects. The overdispersion parameters are also biased to higher values.
\begin{enumerate}
\item Using GenerationCnorm, set 2. \textbf{Note! The random effects where actually completely correlated in this generation, but the covariances to assess bias were said to be zero, so the results are incorrect!}
	\begin{enumerate}
		\item diagREDM: high beta intercept, good beta slope, low lambda, low standard deviation RE
		\item fullREM: high beta intercept, good beta slope, high standard deviation RE, high covariances RE
		\item fullREDM single lambda: high beta intercept, good beta slope, good covariances RE, high lambda, high standard deviation RE
		\item diagREDM single lambda: high beta intercept, good beta slope, high lambda, low standard deviation RE
		\item diagREM: high beta intercept, good beta slope, high standard deviation RE
		\item fullREDM: high beta intercept, good beta slope, good covariances RE, high lambda, low standard deviation RE
	\end{enumerate}
\item Using GenerationCnorm, set 1
	\begin{enumerate}
		\item fullREDM: high beta intercept, good beta slope, good covariances RE, high lambda, low standard deviation RE
\end{enumerate}
\item Using GenerationDMFE1, set 1 and 2, and varying lambda (overdispersion): even in the case where there are no random effects there is a bias in the beta intercept (which is over-estimated). Lambda is too over-estimated. In the second beta set (set 2) not even all the slope betas are unbiased. Set 3 is as bad as set 2
\begin{itemize}
\item With fewer categories ($d=3$): 
\end{itemize}
\end{enumerate}


\newcolumntype{R}[2]{%
    >{\adjustbox{angle=#1,lap=\width-(#2)}\bgroup}%
    l%
    <{\egroup}%
}
\newcommand*\rot{\multicolumn{1}{R{45}{1em}}}% no optional argument here, please!
%\renewcommand*\rot{\multicolumn{1}{R{60}{1em}}}% no optional argument here, please!

%\hspace{-.5in}%\begin{table}[h]
\begin{tabular}{ll|l|cccccccc}
\rot{Dataset} & \rot{Set idx} & \rot{Model (full)} &  \rot{RE structure} &  \rot{Model} &  \rot{Overdispersion structure\footnotemark} & \rot{Beta intercept} & \rot{Beta slope} & \rot{Lambda (overdisp.)} & \rot{Std RE} & \rot{Cov RE}\\\hline
\multirow{7}{*}{GenerationCnorm\footnotemark}&1& fullREDM & fullRE & DM & DL& \hgh & \gd & \hgh & \lw & \gd\\
&2& fullREDM & fullRE & DM & DL &  \hgh & \gd & \hgh & \lw & \gd\\
 & 2 & diagREDM & diagRE & DM & DL& \hgh & \gd & \lw & \lw & -\\
&2& fullREDM SL & fullRE & DM & SL&  \hgh & \gd & \hgh & \hgh & \gd\\
&2& diagREDM SL & diagRE & DM & SL& \hgh & \gd & \hgh & \lw & -\\
&2& fullREM & fullRE & M & - &  \hgh & \gd & - & \hgh & \hgh\\
&2& diagREM & diagRE & M & - & \hgh & \gd & - & \hgh & -\\\hline
\multirow{1}{*}{GenerationCnormdiagRE}&2& fullREDM SL & fullRE & DM & SL& \hgh & \gd & \lw & \hgh & \hgh\\\hline
\multirow{1}{*}{GenerationMGnorm} & 1 & fullREM\\\hline
\multirow{1}{*}{GenerationMGnorm} & 1 & diagREM\\
\end{tabular}
%\end{table}

\footnotetext{DL: Double overdispersion parameter (one for each group). SL: Single overdispersion parameter (shared).}
\footnotetext{Incorrect model! The intercepts are all the same, so there is a perfect correlation! This is not what I said in the assessment of the bias}

\end{document}