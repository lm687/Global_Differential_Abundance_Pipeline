\documentclass{article}

\usepackage{color}

\begin{document}

\section{Notes from the meeting}
\begin{enumerate}
\item make initial values from FE multinomial
\item number of samples, number of categories, zeros, are all things that might explain lack of convergence
\item if there is something that perfectly explains a category, you get error messages too. Make sure that there are no prefect spits
\item it's very sensitive to starting point, so ignoring random effects and just fitting fixed effects is likely to give you a good estimate
\item when it doesn't converge, it means that the problem is that the model is not fitting for the data
\item simulate data. observed in x axis, simulated in y. simulate data under these parameters. rank the observations for lowest to highest, for each sample, and then compare the two. do this for each sample independently. 95\% coverage as area in the line. this is to show that the model is not unrealistic, not that the mpdel os realistic. check the tails
\item the number of degrees of freedom is a function of n and N and are a function of the correlation of the within-patient. if the correlation is very high we have a problem. if the correlation is very low then we are in a better situation
\item fitting something like 22 parameters and 35 samples might be a problem for convergence
\item check with the standard error whether the true values fall in the confidence interval
\item {\color{red} send data to Dom}
\begin{itemize}
	\item worst case scenario: small sample size, high cat
	\item med
	\item best: large number of samples
	\end{itemize}
	(ratio of N/p varying)
\item he's got this method of using some particular initial estimator the estimate, esp. with tricky situations. Estimator found by indirect inference (takes forever, but they found a fast solution). could be applied here. it removes asymptotic bias \& small sample bias
\item in correlated RE it can be very tricky to find a solution
\item {\color{red} send him the algorithm with good/optimized initial parameters (multinom reg)}
\item when you're integrating out the RE, TMB does estimate the random effects, to take them out
\item I use optim, but he doesn't. Instead he uses nlminb instead \\ (\verb| nlminb(start = obj$par, obj = obj$fn, gr = obj$gr)|)
\item when I do opt I already have the information, and sdreport extracts it
\item {\color{red} Share a script in which he can load and run the TMB c function, and then a script which simulates data under the model, and then run code under simulated dataset, and additionally the raw data of the scenarios}
\item simplification might come at a big cost
\item check what they did in the article
\item send him an email about what I thought about the likelihood
\item empirically check unidentifiability due to lack of information: simulation under different number of samples
\item {\color{red} send text about categorical}
\end{enumerate}

\section{Categorical distribution}


\end{document}